\documentclass[12pt]{article}
\usepackage[english,ukrainian]{babel}
\usepackage[letterpaper,top=2cm,bottom=2cm,left=3cm,right=3cm,marginparwidth=1.75cm]{geometry}
\usepackage{amsmath, graphicx, booktabs, listings, xcolor, tcolorbox, lipsum, siunitx, multirow, hyperref, pgfplots, inputenc}
\pgfplotsset{compat=1.17}

\title{\textbf{Криптоаналіз шифру Віженера}}
\author{Варіант 6}
\date{}
    
\begin{document}
\maketitle
\section{Мета роботи}
\quad Засвоєння методів частотного криптоаналізу. Здобуття навичок роботи та аналізу потокових шифрів гамування адитивного типу на прикладі шифру Віженера.
    
\section{Постановка задачі}
\begin{itemize}
    \item обрати вхідйни текст, зашифрувати його за допомогою ключів різної довжини та обрахувати індекси відповідності
    \item розшифрувати вхідний текст згідно варіанту
\end{itemize}
    
\section{Хід роботи}
\quad Для виконання поставленого завдання, після короткого аналізу, я вирішив розбити його на дві різних частини: 
\begin{itemize}
    \item шифрування
    \item розшифрування
\end{itemize}
\quad Труднощі - так як я не був досить уважним при читанні методички, то я спочатку використовував алфавіт довжини 33, що на певний час повністю зупинило мою роботу.
    
\newpage
\subsection{Шифрування}
\quad Маючі такі неперевершені ключі, зашифрувати вхідний текст було досить таким просто.
\begin{table}[h]
\centering
\begin{tabular}{ccc}
\toprule
Довжина ключа & Ключ & індекс відповідності \\
\midrule
2 & оф & 0.0468574 \\
3 & енз & 0.0427849 \\
4 & ивац & 0.0398622 \\
5 & ежпол & 0.03821888 \\
14 & ьськарозвидка & 0.03521032 \\
20 & дайтидокиевазатридня & 0.03481866 \\
\bottomrule
\end{tabular}
\caption{Довжина ключа, ключ, індекс відповідності}
\label{tab:my_table}
\end{table}
    
\quad Теоретичне $I_0$: 0.0303030 \\
    
\quad Теоретичне $I_m$: 0.0561883 \\
    
\quad $I$ для вхідного тексту: 0.0588844 \\
    
Шматок вхідного тексту:
\begin{tcolorbox}[colback=gray!10!white, rounded corners]
    \textit{проблема алкоголизма в нашем обществе становится все более серьезной и насущной алкоголь это не просто напиток это яд который разрушает жизни людей и общества в целом алкоголизм поражает разные}
\end{tcolorbox}\\
    
Шматок вхідного тексту зашифрованого ключем довжини 20:
\begin{tcolorbox}[colback=gray!10!white, rounded corners]
    \textit{урчууйъкирмокоэшпрнвсавчфтпднцфвмсерхтпицсйфщйпшу
    кзсмрпхпсыймнйдыэышсенкхгбыевяосешгцхбшхесищоьоътндооыб
    шячъимтуяачгомфнмлицннцшйюзсщвттюйшорафьцзьхрмопхртцийяр
    дзцонхщшртвсмлчэрдъелалъщмъшцчуоэитыесыгтсы}
\end{tcolorbox}\\
    
\subsection{Дешифрування}
\subsubsection{Визначення довжини ключа}
\noindent
\begin{minipage}{0.5\textwidth}
    \centering
    \begin{tabular}{cc}
    \toprule
    r & індекс \\
    \midrule
    2 &  0.0340552 \\
    3 &  0.0340833 \\
    4 &  0.0340909 \\
    5 &  0.0341231 \\
    6 &  0.0339760 \\
    7 &  0.0341585 \\
    8 &  0.0340264 \\
    9 &  0.0339186 \\
    10 & 0.0339288 \\
    \bottomrule
    \end{tabular}
\end{minipage}%
\begin{minipage}{0.5\textwidth}
    \centering
    \begin{tabular}{cc}
    \toprule
    r & індекс \\
    \midrule
    11 & 0.0340091 \\
    12 & 0.0340577 \\
    13 & 0.0340373 \\
    14 & 0.0340840 \\
    15 & 0.0342267 \\
    16 & 0.0340255 \\
    17 & 0.0555077 \\
    18 & 0.0338367 \\
    19 & 0.0339287 \\
    \bottomrule
    \end{tabular}
\end{minipage} \\
    
\quad Очікуване значення індексу: 0.0561860, звідси робимо висновок, що довжина ключа 17.
    
\subsubsection{Одержані ключі}
\quad Значення ключа, одержане шляхом співставлення найчастіших літер блоків
найчастішій літері мови: \textbf{возврааениеджинда}. Значення ключа, одержане із використанням функції $M_i(g)$: \textbf{возвращениеджинна}. Бачимо, що ключ отриманий за допогою методу частот трошки відрзняється від того, що ми отримали за допомогою функції $M_i(g)$ і на відміну від $M_i(g)$, він не має логічного значення. Тому скорегуємо його до виду: \textbf{возвращениеджинна}.\\
    
\noindent
Шматок зашифрованого тексту:
\begin{tcolorbox}[colback=gray!10!white, rounded corners]
    \textit{жьчрдеврйкужояьхвфьчэъоъашгтмцифавицопшнюфытнижуфтмнцьрвяихыон
    пщотоонкязиекчхмкхеъхшефюзгютщрьшуфжйыщсфюхкведбъцооффьннкцлрьокчэцо
    жыиэйкррмуводнгнзоцихъынмикыпзхийеыоъйюдтбоюпмбтнцмйцивэоеофюбкзиыт
    хдепндетахлуойусизяцижхввщфвфартышыжщячеррхышинхатчяицюьифййвывжшчц
    здицяаейфзфмзщфэнййсгэыдпьрдърщнъгтйсжохлпушоютйдъизтнфыунрящктсыд
    фрцхфпсннкууеыоъешдттпщтяиоущтюпзжикецвхншюгьрсыажкянцтсхтднрчшкб
    тюсирйдмнфнезэчзфедещрьцфчысвкстрхгзцылрдчряйсбызяъсгшэщнвхцшанзьфкб
    аетткцтчъымнкциэыолз}
\end{tcolorbox}
\newpage
Шматок розшифрованого тексту: 
\begin{tcolorbox}[colback=gray!10!white, rounded corners]
    \textit{радуясьвозможностиразмятьсягурьбойнаправилиськобрывуперебрасываясь
    шуточкамиидурачасьвнихигралащенячьяэнергиямолодостиидорофейльвовичнамгновениепо
    завидовалзадоруиоптимизмуюношейидевушекгодящихсяемучутьлиневовнукионтоже
    полюбовалсянаснежнобелыйкуполвтрехкилометрахотобрывапотомтихонько
    отошелотрезвящихсямолодыхлюдейипрошелсявдольобрывавглядываясьв
    противоположнуюстенуущельявзгляднаткнулсянарядчерныхотверстийпохожих
    наследыпулеметнойочередизаинтересовавшисьдорофейльвовичпрыгнулвниз
    ивключивантигравпересекущельеопустилсянаузкийкарнизпередсамойбольшойдыройо
    предупреждениигиданеотходитьдалекоотфлайтаонзабылдыраоказаласьвходомвпещеру}
\end{tcolorbox}
    
\section{Висновок}
    
\quad З результатів бачимо, що в даній реалізації програми визначення символів ключа за допомогою методу $M_i(g)$ дало більш коректний результат порівняно з методом частот, що жодним чином не порушує наші теоретичні знання в межах даної задачі. 
\end{document}