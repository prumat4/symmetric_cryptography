\documentclass[12pt]{article}
\usepackage[english,ukrainian]{babel}
\usepackage[letterpaper,top=2cm,bottom=2cm,left=3cm,right=3cm,marginparwidth=1.75cm]{geometry}
\usepackage{amsmath, graphicx, booktabs, listings, xcolor, tcolorbox, lipsum, siunitx, multirow, hyperref, pgfplots, inputenc}
\pgfplotsset{compat=1.17}

\title{\textbf{Криптоаналіз афінної біграмної підстановки}}
\author{Варіант 6}
\date{}
    
\begin{document}
\maketitle
\section{Мета роботи}
\quad Набуття навичок частотного аналізу на прикладі розкриття моноалфавітної підстановки; опанування прийомами роботи в модулярній арифметиці.
        
    
\subsection{Дешифрування}
\subsubsection{Найчастіші біграми шифротексту}
Найчастішими біграма в заданому шифротексті виявились ось такі: ['цл', 'ял', 'ае', 'ле', 'чо'], звідси зрозуміло, що текст не є змістовним, бо такі біграми в ньому не зустрічались би.

\subsection{Генерація ключів}
\quad Рахуємо найчастіші біграми шифротексту і беремо найчастіші біграми мови, конвертуємо їх в числові значення, 
розв'язуємо систему, це все бахується на тому на такому факті: афінний шифр зберігає статистичні властивості мови, пов’язані із частотами біграм, маємо потенційні а та b.
Далі для кожного з них, дешифруємо вхідний зашифрований текст, і перевіряємо його на змістовність.

\subsubsection{Критерій змістовності тексту}
\quad Як ми це робимо, рахуємо частоти усіх символів розшифрованого тексту, тут у нас є ось цей ranksize - те скільки найчастіших і найрідших символів ми беремо, потім перевіряємо чи мають ці top i bottom chars найпопулярніші і найрідші символи мови відповідно, ну і відповідно регуляємо оцінку, якщо вони є або немає, потім також перевіряємо індекс відповідності.

\newpage
\subsection{Результати}
Шматок зашифрованого тексту:
\begin{tcolorbox}[colback=gray!10!white, rounded corners]
\textit{йжтаеэщащбаеыбзэхечоетульйсулцтюаьебхсмжзаюэфйжлнэцтклиувь
зэлцюедкйетлофгбйбфлаэжугосрчусфашолыьзатййуыйвичьдмэбдцялшаиуошул
обяфьбацкфщмюэзыкюццкфлеисядыфрцксчоюйрлщегмююзаяййугугфклиуулиулц
оюфюхевюфйвеаесачсчопчцлхулбщлербноулехебрбннллйжшбцвбошьййшктгбаз
ошоффйжлнэеэажкюмиуэфщшцюйюэщйщлгшеэыэнзрцшцчлвгйтхйщлхэзывгмжуэбо
аанафщлйрзажбщйрмфллжлпфцлуичьтфшцюйлфгййшцлпаюеюэбщзазяйрлцфунбсф
хаечыэнзхоцжсаыитсольймйсфолкцулхзобнцзеасвеелгйхьечццщюхеьащмцжбщ
юйзльйщбфлбиоптиилвбцьдмэбьтофлйжлмллакнцлцщебдасццйийфлципрулхноц
ьлцлеузбзитснозэымновцлфцлчеебшуустиофоббэжфллгувешцщлрэелещянхеза
вцлэяйжлгйюйулэйбэымнлещянхекскеаелеыизаьтвбшабцллйшгбцьдмэбтыпаль
аозаопкечодпебцфилхнзаююагаечявафщцжчьфщжйфллекюдтрййувьцлйубисасм
хешщиежцьюцжяаццдэйщебфьлщвьопцлсяпаусхлдцисаетиййбиююаьюеэропебчэ
фюжлвлмфчлхмтивьтеаехйшйжштйийвьцлаешифюыэтйшйхуьсоцялшащбнфвлллощ
}
\end{tcolorbox}

Шматок розшифрованого тексту: 
\begin{tcolorbox}[colback=gray!10!white, rounded corners]
\textit{атызнаешьсколькоразмывэтомгодуиграливбейсболавпрошломавпоза
прошломнистогониссегоспросилтомгубыегодвигалисьбыстробыстроявсезапи
салтысячпятьсотшестьдесятвосемьразасколькоразячистилзубызадесятьлет
жизнишестьтысячразарукимылпятнадцатьтысячразспалчетыреслишнимтысячи
разиэтотольконочьюиселшестьсотперсиковивосемьсотяблокагрушвсегодвес
тиянеоченьтолюблюгрушичтохочешьспросиуменявсезаписаноесливспомнитьи
сосчитатьчтояделалзавседесятьлетпрямотысячимиллионовполучаютсявотво
тдумалдугласопятьоноближепочемупотомучтотомболтаетноразведеловтомео
нвсетрещититрещитсполнымртомотецсидитмолчанасторожилсякакрысьатомвс
еболтаетникакнеугомонитсяшипитипенитсякаксифонссодовойкнигяпрочелче
тыресташтуккиносмотрелитогобольшесорокфильмовсучастиембакаджонсатри
дцатьсджекомхоксисорокпятьстомоммиксомтридцатьдевятьсхутомгибсономс
тодевяностодвамультипликационныхпрокотафеликсадесятьсдугласомфербен
}
\end{tcolorbox}
    
\section{Висновок}
    
\quad В результаті роботи програми, вдалось успішно дешифрувати вхідний текст, а тому обраний кретерій змістовності тексту є коректним, та його можна в подальшому використовувати. 
\end{document}
